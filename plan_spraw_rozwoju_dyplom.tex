%% -*- coding: utf-8 -*-
\documentclass[a4paper,titlepage,13pt,draft]{mwart}
\setlength{\textheight}{22cm}
\setlength{\textwidth}{18cm}
\setlength{\oddsidemargin}{0.1cm}
\setlength{\evensidemargin}{0.1cm}
\usepackage[english,polish]{babel}
\usepackage[OT1]{fontenc}
\usepackage[utf8]{inputenc}
%\usepackage{iwona}
\usepackage{polski}
\frenchspacing
\usepackage{indentfirst}
\title{\textsc{\huge{Plan Rozwoju nauczyciela mianowanego ubiegającego się\\
o stopień awansu zawodowego nauczyciela dyplomowanego}}}
\author{Adam Bojanowski}
\date{1 września 2014}
%\selectlanguage{english}
\usepackage[polish]{babel}
%\usepackage[OT4]{fontenc}
%\usepackage[utf8]{inputenc}
%\usepackage{iwona}
%\usepackage{polski}
\pagestyle{headings}%{empty}
\begin{document}
\maketitle
%\bibliographystyle
%\textsc{\huge{Plan Rozwoju nauczyciela mianowanego ubiegającego się 
%o stopień awansu zawodowego nauczyciela dyplomowanego}}
\begin{center}
%\begin{minipage}{10cm}
\title{\textsc{\Large{Plan rozwoju zawodowego\\mgr Adama Bojanowskiego\\nauczyciela przedmiotów zawodowych\\i~języka angielskiego zawodowego\\w Technikum Łączności\\w Zespole Szkół nr. 37 im. A. Osieckiej\\w Warszawie}}}

%\end{minipage}\\
%24 września 2014
\end{center}
\section*{Zobowiązania ogólne nauczyciela w okresie trwania stażu:}
\subsection*{Opracowanie i~wdrożenie programu działań edukacyjnych służących podniesieniu oferty Szkoły tj. wdrożenie kursu programowania w~języku Python --- oferta dla klas 1 -- 3 technikum.}
\subsection*{Podnoszenie kwalifikacji~nauczyciela --- uzyskanie umiejętności~posługiwania się drugim językiem obcym -- hiszpańskim na poziomie zaawansowanym.}
\subsection*{Wynikających z planu rozwoju Szkoły wobec oczekiwań rynku pracy: intensyfikacja nauki~języka angielskiego zawodowego, praca z~uczniem zdolnym, przygotowanie do~egzaminów zewnętrznych.}
\end{center}
%\begin{tabular}{|l|l|l|l|l|} %5 col;vertical lines
%\section{Priorytety w rozwoju zawodowym} %& zadania & forma realizacji~& termin & uwagi
%\begin{tabular*}{\textwidth}%
%{@{\extracolsep{\stretch{1}}}|l|r|r|r|r|r|}\hline
%Standardy na podst.
%Rozporządzenia MEN z dn. 3.08.2000
%/pozycja w Rozp. MEN/
%& \multicolumn{2}{|c|}{Wys. w~m~n.p.m.} &
%Długość & \multicolumn{2}{|c|}{Nachylenie \%}\\
%\cline{2-3} \cline{5-6}% \cline można wstawiać wielokrotnie
%& początek & koniec & w~km & śr. & max
%\\ \hline
%Col du Galibier & 1401 & 2646 & 18,1 & 6,9\% & 14,5 \\
%Alpe D’Huez
%& 724 & 1815 & 14,2 & 7,7\% & 15,0 \\
%Passo Gavia
%& 1734 & 2618 & 13,5 & 6,5\% & 20,0 \\ \hline
%\end{tabular*}
\newpage
\center
\begin{tabular}{ | p{2.4cm} | p{2.5cm} | p{5cm} | p{1.4cm} | p{2cm} | p{2.1cm} |}
\hline
Standardy na podst. Rozporządzenia MEN z dn. 3.08.2000 (pozycja w Rozp. MEN) & Wskaźniki~ogólne dotyczące realizacji~standardu & Wskaźniki~szczegółowe dotyczące realizacji~standardu & Termin realizacji~& Konsultanci, instytucje i~osoby wspierające & Dowody realizacji, uwagi~\\ \hline \hline
\multirow{Zapoznanie się z aktami~prawnymi} & Poznanie procedury awansu zawodowego na stopień nauczyciela dyplomowanego & Analiza przepisów prawa oświatowego dotyczących awansu zawodowego (Ustawa Karta Nauczyciela, Ustawa o systemie światy, Rozporządzenie MEN z 3.08. 2000r.).\newline Zapoznanie się z materiałami~publikowanymi~na stronach internetowych: CODN, MEN, Eduseek, Edukator. & wrzesień 2014 & dyrekcja szkoły & poprawnie sformułowany wniosek o rozpoczęcie stażu, plan rozwoju zawodowego i~rozpoczęte sprawozdanie z jego realizacji. \\ \cline{2-6} 
& Znajomość przepisów prawa oświatowego & Samodzielne śledzenie zmian w prawie oświatowym\newline
Stosowanie się do zmian w przepisach prawa & okres stażu & dyrekcja szkoły & notatki~własne \\ \cline{2-6}
& Przypomnienie zasad funkcjonowania i~organizacji~zadań szkoły. & Analiza dokumentacji:\newline
statutu,\newline 
programu rozwoju szkoły,\newline
wewnątrzszkolnego systemu oceniania,\newline
planu dydaktyczno – wychowawczego szkoły. & wrzesień 2014 & dyrekcja szkoły & notatki~własne \\ \hline
\multirow{Podejmowanie działań mających na celu doskonalenie warsztatu i~metod pracy (§5 ust.1 pkt.1)} & Ocena własnych umiejętności~&
Autoewaluacja kompetencji~\newline 
Opracowanie planu rozwoju zawodowego & wrzesień 2014 & dyrekcja szkoły & plan rozwoju zawodowego \\ \cline{2-6}
& Dokumento\-wa\-nie realizacji~planu rozwoju. & Gromadzenie dokumentów: świadectw, zaświadczeń, zdjęć\newline
Sporządzanie sprawozdań, notatek. & okres stażu & dyrekcja szkoły, współpracownicy, uczniowie, instytucje & świadectwa, zaświadczenia, zdjęcia, sprawozda\-nia.\\ \cline{2-6}
& Przygotowanie sprawozdania z realizacji~planu rozwoju zawodowego. & Autorefleksja, autoanaliza.\newline
Opis realizacji~planu rozwoju. & maj -- czerwiec 2017 & dyrekcja szkoły & sprawozdanie z~realizacji~planu rozwoju zawodowego \\ \cline{2-6}
& Zakończenie realizacji~planu rozwoju zawodowego. & Analiza dokumentów.\newline
Sprawdzenie materiałów pod kątem kolejności~realizacji~oraz poprawności~merytorycznej.\newline Sprawdzenie kompletności~dokumentacji~zgodnie z Rozporządzeniem MEN z dnia 3.08. 2000r. & czerwiec 2017 & dyrekcja szkoły, koleżanki~i~koledzy z pracy & poprawnie sformułowany wniosek o podjęcie postępowania kwalifikacyjnego, dokumentacja formalna \\ \hline
\end{tabular}
\newpage
\begin{tabular}{ | p{2.4cm} | p{2.5cm} | p{5cm} | p{1.4cm} | p{2cm} | p{2.1cm} |}
\hline
Standardy na podst. Rozporządzenia MEN z dn. 3.08.2000 (pozycja w Rozp. MEN) & Wskaźniki~ogólne dotyczące realizacji~standardu & Wskaźniki~szczegółowe dotyczące realizacji~standardu & Termin realizacji~& Konsultanci, instytucje i~osoby wspierające & Dowody realizacji, uwagi~\\ \hline \hline
\multiline{Podejmowanie działań mających na celu doskonalenie warsztatu i~metod pracy c.d.}
& Prowadzenie dokumentacji~szkolnej zgodnie z przepisami~& 
Analiza autorskiego nauczania języka angielskiego zawodowego JAZ\newline
Opracowanie rozkładu materiału na dany rok szkolny\newline
Systematyczne zapisy w dziennikach lekcyjnych\newline
Prowadzenie autorskich zajęć pozalekcyjnych\newline
Prowadzenie dziennika zajęć pozalekcyjnych\newline & okres stażu & dyrekcja szkoły, koleżanki~i~koledzy & 
program nauczania JAZ, rozkłady materiału, zapisy w dokumentacji~szkolnej\\ \cline{2-6}
& Organizacja zaplecza pracy & Bieżąca opieka nad Pracownią Systemów Operacyjnych\/Komputerowych.\newline
Unowocześnienie, uzupełnienie wyposażenia pracowni.\newline
Tworzenie nowych pomocy dydaktycznych & okres stażu & dyrekcja szkoły, koleżanki~i~koledzy, uczniowie & zdjęcia, notatka własna\\ \cline{2-6}
& Weryfikacja nabytej wiedzy i~umiejętności~uczniów & Wypracowanie różnorodnych form oceny wiedzy i~umiejętności~nabytych przez uczniów w dziedzinach przedmiotów informatycznych oraz JAZ\newline
Opracowanie arkuszy oceny poszczególnych sprawności~językowych JAZ\newline
Opracowanie metodyki~przygotowywania ćwiczeń i~oceniania przedmiotów zawodowych & okres stażu & koleżanki~i~koledzy & arkusze oceny, testy, sprawdziany, wnioski~\\ \cline{2-6}
& Poznanie nowych trendów i~rozwiązań metodycznych & Udział w kursach i~warsztatach metodycznych\newline
Udział w lekcjach otwartych prowadzonych przez nauczycieli\newline
Lektura własna\newline
Uzyskanie dodatkowych kwalifikacji~& staż & dyrekcja szkoły, instytucje, koleżanki~i~koledzy & sprawozda\-nia, opisy, notatki~\\ \cline{2-6}
& Wykorzystanie metod aktywizujących & Zapoznanie się z metodami~aktywizującymi: szkolenia, lektura.\newline
Obserwacja stosowania metod aktywizujących przez innych nauczycieli.\newline
Wdrażanie metod aktywizujących w pracy z uczniami. & staż & dyrekcja szkoły, koleżanki~i~koledzy, uczniowie & sprawozda\-nia, notatki, opisy, konspekty lekcji~\\ \cline{2-6}
& Sprawozdanie z realizacji~podjętych zadań & Analiza podjętych zadań\newline
Przygoto\-wa\-nie sprawozdania & maj -- czerwiec 2017 & dyrekcja szkoły & sprawozdanie \\ \hline
\end{tabular}
\newpage
\begin{tabular}{ | p{2.4cm} | p{2.5cm} | p{5cm} | p{1.4cm} | p{2cm} | p{2.1cm} |}
\hline
Standardy na podst. Rozporządzenia MEN z dn. 3.08.2000 (pozycja w Rozp. MEN) & Wskaźniki~ogólne dotyczące realizacji~standardu & Wskaźniki~szczegółowe dotyczące realizacji~standardu & Termin realizacji~& Konsultanci, instytucje i~osoby wspierające & Dowody realizacji, uwagi~\\ \hline \hline
\multiline{Uczestniczenie w realizacji~zadań wykraczających poza wykonywane obowiązki~służbowe\newline
/§ 5 ust. 1 pkt. 2/}&Organizacja życia klasy i~szkoły&Organizacja wycieczek klasowych i~szkolnych\newline
Wyjścia z uczniami~na wystawy, np. Automaticon,\newline
Organizacja konkursów wiedzy\newline
Przygotowanie materiałów szkoleniowych&okres stażu&dyrekcja szkoły, koleżanki~i~koledzy, rodzice, uczniowie&
programy wyjść i~wycieczek, zdjęcia, sprawozda\-nia, potwierdzenia, materiały szkoleniowe\\ \cline{2-6}
&Nagradzanie szczególnych osiągnięć uczniów w dziedzinie przedmiotów zawodowych i~JAZ& Przygotowanie dyplomów uznania.\newline Zakup nagród za szczególne osiągnięcia uczniów.\newline Uroczyste wręczenie nagród i~dyplomów&okres stażu&dyrekcja szkoły, koleżanki~i~koledzy&wzorzec dyplomu, potwierdzenia\\ \cline{2-6}
&Integrowanie rodziców ze szkołą&Regularne spotkania z rodzicami~w ramach konsultacji\newline
Budzenie zainteresowania ofertą szkoły&okres stażu&dyrekcja szkoły, koleżanki, koledzy, psycholog i~pedagog&notatki~ze spotkań\\ \cline{2-6}
&Sprawozdanie z realizacji~podjętych zadań&Analiza podjętych zadań&Przygotowanie sprawozdania&maj -- czerwiec 2017&dyrekcja szkoły, sprawozdanie\\ \hline
\multiline{Samodzielnie lub poprzez udział w formach doskonalenia zawodowego, pogłębianie swojej wiedzy oraz umiejętności~służące własnemu rozwojowi~oraz poniesieniu poziomu pracy szkoły -- miejsca zatrudnienia\newline
(§ 5 ust. 1 pkt. 3)}&Uczestniczenie w pracach organów szkoły związanych z realizacją jej funkcji~i~wynikających z nich zadań.&Współpraca z dyrektorem szkoły, wychowawcami, pedagogiem szkolnym.\newline Uczestniczenie w posiedzeniach Rady Pedagogicznej&okres stażu&dyrekcja szkoły, koleżanki~i~koledzy, psycholog i~pedagog&notatki~własne, zaświadczenia dyrektora\\ \hline
\end{tabular}
\newpage
\begin{tabular}{ | p{2.4cm} | p{2.5cm} | p{5cm} | p{1.4cm} | p{2cm} | p{2.1cm} |}
\hline
Standardy na podst. Rozporządzenia MEN z dn. 3.08.2000 (pozycja w Rozp. MEN) & Wskaźniki~ogólne dotyczące realizacji~standardu & Wskaźniki~szczegółowe dotyczące realizacji~standardu & Termin realizacji~& Konsultanci, instytucje i~osoby wspierające & Dowody realizacji, uwagi~\\ \hline \hline
&Poszerzenie wiedzy i~umiejętności&Udział w szkoleniach organizowanych wewnątrz szkoły\newline
Udział w kursach i~szkoleniach organizowanych przez instytucje pozaszkolne adekwatnych do potrzeb własnych i~szkoły&okres stażu&dyrekcja szkoły, koleżanki~i~koledzy, psycholog i~pedagog&notatki~własne, zaświadczenia, świadectwa\\ \cline{2-6}
&Uzyskanie dodatkowych kwalifikacji&Uzyskanie dodatkowych kwalifikacji~wynikających z potrzeb własnych\newline
Uzyskanie dodatkowych kwalifikacji~wynikających ze specyfiki~i~potrzeb szkoły&okres stażu&organiza\-to\-rzy szkoleń, dyrekcja szkoły&zaświad\-cze\-nia, notatki\\ \cline{2-6}
&Samodzielne studiowanie literatury pedagogicznej i~metodycznej, wymiana doświadczeń, poszukiwanie rozwiązań&Śledzenie rynku wydawniczego: czasopism, książek\newline
Analiza wybranych pozycji~  z literatury pedagogicznej i~metodycznej.\newline
Udział w spotkaniach zespołów nauczycielskich.\newline
Udział w posiedzeniach Rady Pedagogicznej.\newline
&staż&dyrekcja szkoły, koleżanki~i~koledzy&bibliografia, własne notatki, wnioski, sprawozdania\\ \cline{2-6}
&Sprawozdanie z realizacji~podjętych zadań&Analiza podjętych zadań\newline Przygotowanie sprawozdania&maj -- czerwiec 2017&dyrekcja szkoły&sprawozdanie\\ \hline
\multiline{Opracowanie i~wdrażanie przedsięwzięć i~programów na rzecz doskonalenia swojej pracy i~podwyższania jakości~pracy szkoły, w tym wykorzystywanie i~doskonalenie umiejętności~stosowania technologii~komputerowej i~informacyjnej.\newline
(§ 5 ust. 2 pkt. 1,\newline 
§ 7 ust. 4 pkt. 1 i~4)}
&Doskonalenie swojej pracy i~podwyższanie jakości~pracy szkoły&
Ewaluacja i~wdrażanie wewnątrzszkolnego systemu oceniania.\newline
Opracowanie i~wdrażanie przedmiotowego systemu oceniania.\newline
Praca w zespołach przedmiotowych: przedmiotów zawodowych i~języków obcych.\newline
Praca w zespołach nauczycieli~uczących w danej klasie.
&staż&dyrekcja szkoły, koleżanki~i~koledzy&
przedmiotowy system oceniania z języka angielskiego zawodowego JAZ, notatki, sprawozda\-nia\\ \hline
\end{tabular}
\newpage
\begin{tabular}{ | p{2.4cm} | p{2.5cm} | p{5cm} | p{1.4cm} | p{2cm} | p{2.1cm} |}
\hline
Standardy na podst. Rozporządzenia MEN z dn. 3.08.2000 (pozycja w Rozp. MEN) & Wskaźniki~ogólne dotyczące realizacji~standardu & Wskaźniki~szczegółowe dotyczące realizacji~standardu & Termin realizacji~& Konsultanci, instytucje i~osoby wspierające & Dowody realizacji, uwagi~\\ \hline \hline
\multiline{c.d.}&Wykorzystanie technologii~komputerowej i~informacyjnej w pracy pedagogicznej i~dydaktycznej.&Wykorzystanie komputerowych programów edukacyjnych, Internetu, encyklopedii~multimedialnych w pogłębianiu wiedzy własnej i~ucznia.\newline
Projekt i~wykonanie pomocy dydaktycznych.\newline
Opracowanie testów, sprawdzianów.\newline
Prowadzenie zajęć z wykorzystaniem komputerowych programów edukacyjnych oraz Internetu.
&staż&dyrekcja szkoły, koleżanki~i~koledzy&przykładowe materiały z Internetu wykorzystane na lekcjach, zdjęcia, projekty, pomoce dydaktyczne,testy, sprawdziany, konspekty lekcji\\ \cline{2-6}
&Promowanie szkoły&Udział w konkursach i~olimpiadach przedmiotowych.\newline
Organizacja konkursów o lokalnym i~szerszym zasięgu.&staż&dyrekcja szkoły, koleżanki~i~koledzy&zaświadczenia, dyplomy, programy\\ \cline{2-6}
&Opis i~analiza podjętych działań&Dokonanie przeglądu podjętych działań\newline
Wykonanie opisu i~analizy podjętych działań&maj -- czerwiec 2017&dyrekcja szkoły&opis i~analiza podjętych działań\\ \hline
\multiline{Umiejętność dzielenia się swoją wiedzą i~doświadczeniem z~innymi~pracownikami~szkoły\newline
(§ 5 ust. 2 pkt. 2,\newline
§ 7 ust. 4 pkt. 1 i~4)}
&Pełnienie dodatkowych funkcji~w szkole&Pełnienie funkcji~opiekuna stażu&okres stażu&dyrekcja szkoły, koleżanki~i~koledzy&kontrakty, zaświadczenia\\ \cline{2-6}
&Opis i~analiza podjętych działań&Dokonanie przeglądu podjętych działań\newline
Wykonanie opisu i~analizy podjętych działań&maj -- czerwiec 2017&dyrekcja szkoły&opis i~analiza podjętych działań\\ \hline
\multiline{
Opracowanie i~wdrożenie programu dotyczącego działań edukacyjnych, wychowawczych, opiekuńczych lub innych związanych z~oświatą, pomocą społeczną\newline
(§ 5 ust. 2 pkt. 3a,\newline
§ 7 ust. 4 pkt. 2 i~4)}
&Opracowywanie i~wdrożenie własnych projektów dydaktyczno – wychowawczych.&
Zorganizowanie dla młodzieży zajęć pozalekcyjnych „RoboTy: projektowanie i~użytkowanie"\newline,,Python - uniwersalny język programowania wszystkiego"\newline
Opracowanie własnego programu nauczania j.~angielskiego zawodowego\newline
Opracowanie programów konkursów przedmiotowych&okres stażu&dyrekcja szkoły, koleżanki~i~koledzy&programy zajęć pozalekcyjnych, konspekty zajęć, programy konkursów\\ \hline
\end{tabular}
\newpage
\begin{tabular}{ | p{2.4cm} | p{2.5cm} | p{5cm} | p{1.4cm} | p{2cm} | p{2.1cm} |}
\hline
Standardy na podst. Rozporządzenia MEN z dn. 3.08.2000 (pozycja w Rozp. MEN) & Wskaźniki~ogólne dotyczące realizacji~standardu & Wskaźniki~szczegółowe dotyczące realizacji~standardu & Termin realizacji~& Konsultanci, instytucje i~osoby wspierające & Dowody realizacji, uwagi~\\ \hline \hline
\multiline{Opracowywanie i~wdrożenie c.d.}
&Opis i~analiza podjętych działań&Dokonanie przeglądu podjętych działań\newline
Wykonanie opisu i~analizy podjętych działań&maj -- czerwiec 2017&dyrekcja szkoły&opis i~analiza podjętych działań\\ \hline
          
\multiline{Opracowanie co najmniej dwóch publikacji, referatów lub innych materiałów związanych z~wykonywaną pracą i~opublikowanie ich, wygłoszenie lub popularyzację w~innej formie
(§ 5 ust. 2 pkt. 3b,\newline
§ 7 ust. 4 pkt. 2 i~4)}
&Przygotowanie i~popularyzacja opracowań własnych&Przygotowanie materiałów nt. Awansu Zawodowego i~przedstawienie ich na posiedzeniu Rady Pedagogicznej.\newline
Przygotowanie innych materiałów szkoleniowych i~przeprowadzenie szkoleń.\newline
Przygotowanie referatów i~ich wygłoszenie.\newline
Publikacja własnych referatów, opracowanych materiałów, scenariuszy lekcji~w bibliotece szkolnej oraz, w miarę możliwości, w czasopismach specjalistycznych oraz na stronach internetowych poświęconych edukacji.&okres stażu&dyrekcja, koleżanki~i~koledzy, uczniowie&
opracowane i~opublikowane materiały i~referaty, zaświadczenia\\ \cline{2-6}
&Opis i~analiza podjętych działań&Dokonanie przeglądu podjętych działań\newline
Wykonanie opisu i~analizy podjętych działań&maj -- czerwiec 2017&dyrekcja szkoły&opis i~analiza podjętych działań\\ \hline
\multiline{
Prowadzenie otwartych zajęć, w szczególności~dla nauczycieli~stażystów i~nauczycieli~kontraktowych lub podejmowanie działań związanych z~wewnątrzszkolnym doskonaleniem zawodowym
(§ 5 ust. 2 pkt. 3c,\newline
§ 7 ust. 4 pkt. 2 i~4)}
&Prowadzenie otwartych zajęć&Przygotowanie otwartych zajęć\newline
Przeprowadzenie otwartych zajęć&okres stażu&dyrekcja, koleżanki~i~koledzy, uczniowie&
konspekty, wnioski~z~obserwacji, omówienie zajęć\\ \cline{2-6}
&Organizowanie form wewnątrzszkolnego doskonalenia&Przygotowanie form szkolenia wynikających z bieżących potrzeb.\newline
Umieszczenie materiałów w bibliotece szkolnej.\newline
Pełnienie roli~opiekuna praktyk pedagogicznych i~opiekuna stażystów.\newline
Obserwacja zajęć prowadzonych przez praktykantów, stażystów.&okres stażu&dyrekcja, koleżanki~i~koledzy, uczniowie&
zaświadczenie, materiały szkoleniowe, potwierdzenie praktyk,
kontrakty, potwierdzenie dyrektora szkoły\\ \hline









\end{tabular}

\begin{flushleft}
\section{Zapoznanie się z aktami~prawnymi~dotyczącymi~procedury awansu}
\subsection{Analiza przepisów prawa oświatowego dotyczących awansu zawodowego (Karta Nauczyciela, Ustawa o systemie światy, Rozporządzenie MEN z 3.08.2000r.).}
\subsection{Zapoznanie się z materiałami~publikowanymi~przez: CODN, MEN, Eduseek, Edukator.}
\subsection{Znajomość przepisów prawa oświatowego.}
\subsubsection{Samodzielne śledzenie zmian w prawie oświatowym i~stosowanie się do zmian.}
\subsection{Przypomnienie zasad funkcjonowania i~organizacji~zadań szkoły.}
\subsubsection{Analiza dokumentacji: statutu, programu rozwoju szkoły, wewnątrzszkolnego systemu oceniania, planu dydaktyczno – wychowawczego szkoły.}
\section{Doskonalenie warsztatu i~metod pracy.} \label{doskonalenie}
\subsection{Ocena własnych umiejętności.}
\subsubsection{Autoewaluacja kompetencji.}
\subsubsection{Opracowanie planu rozwoju zawodowego.}
\subsection{Organizacja zaplecza pracy.}
\subsubsection{Bieżąca opieka nad pracownią systemów operacyjnych i~komputerowych.}
\subsubsection{Unowocześnienie, uzupełnienie wyposażenia pracowni~systemów operacyjnych i~komputerowych}.
\subsubsection{Tworzenie i~modyfikacja pomocy dydaktycznych.}

\subsection{Weryfikacja nabytej wiedzy i~umiejętności~uczniów.}
\subsubsection{Wypracowanie różnorodnych form oceny wiedzy i~umiejętności~nabytych przez uczniów w dziedzinie przedmiotów zawodowych i~języka angielskiego zawodowego.}
\subsubsection{Opracowanie i~wprowadzenie metod automatycznego testowania.}\label{autotest}
\subsection{Poszerzenie wiedzy i~umiejętności.}
\subsubsection{Lektura i~studia własne nad przedmiotem.}
\section{Dzieleni~się wiedzą i~doświadczeniem z innymi~nauczycielami.}
\subsection{Pełnienie dodatkowych funkcji~w szkole.}\label{funkcje}
\subsubsection{Pełnienie funkcji~opiekuna stażu.}\label{opiekun}
\subsubsection{Pełnienie funkcji~administratora dziennika internetowego.}
\section{Opracowanie i~wdrożenie programu dotyczącego działań edukacyjnych.}
\subsection{Opracowywanie i~wdrożenie własnych projektów dydaktycznych.} \label{dzeduk}
\subsubsection{Zorganizowanie dla młodzieży zajęć pozalekcyjnych „Język programowania Python dla początkujących".}\label{kursPython}
\subsection{Przygotowanie i~popularyzacja opracowań własnych.}
\subsubsection{Przygotowanie materiałów szkoleniowych i~przeprowadzenie szkoleń. Przygotowanie referatów i~ich wygłoszenie. Publikacja własnych referatów, opracowanych materiałów, scenariuszy lekcji~w bibliotece szkolnej oraz, w miarę możliwości, w czasopismach specjalistycznych oraz na stronach internetowych poświęconych edukacji.}
\newpage
\subsection{Wykonanie przeglądu i~analizy podjętych działań.}
Działania podejmowane przeze mnie w okresie stażu obejmowały wiele obszarów, w których wyróżnić należy dwa \textit{meta} oraz wiele wzajemnie przenikających się podobszarów:
\begin{enumerate}
\item{JAZ tj. języka angielskiego zawodowego} oraz 
\item{Pracowni Systemów Operacyjnych}\newline W tym drugim wyróżnić można kilka jak: \begin{enumerate}
\item{Pracownia SO (Linux Debian)},
\item{Pracownia Lokalnych Sieci Komputerowych (usługi serwera sieciowego Linux SUSE Enterprise Server)},
\item{Pracownia Systemów Komputerowych (Linux Debian)},
\end{enumerate}
\item{Istotnym obszarem, pochłaniającym dużą ilość czasu -- około 120 godzin zegarowych w drugim semestrze\footnote{\url{https://synergia.librus.pl/dodatkowe\_godziny}-- dnia 10-06-2017~r.} -- stały się także zajęcia dodatkowe zatytułowane \textit{RoboTY} w założeniu poświęcone rozwijaniu zainteresowań politechnicznych młodzieży w naszej Szkole.}
\item{Pełniłem także dodatkową funkcję administratora dziennikiem internetowym w okresie wrzesień 2015 - grudzień 2016 zgodnie z \ref{funkcje} wymienionym na stronie \pageref{funkcje} \textit{Planu Rozwoju}}.
\item{Sprawowałem także opiekę nad stażystą realizującym staż na nauczyciela mianowanego prof. Jackiem Paszkiewiczem\footnote{\url{/home/navegante/Projekt oceny dorobku zawodowego za okres stażu.pdf}} zgodnie z \ref{opiekun}} wymienionym na stronie \pageref{opiekun} \textit{Planu Rozwoju}}.
\item{Doskonalenie warsztatu i metod pracy} to kolejny obszar, wpływający na pozostałe, jak i z nich wynikający}.
\item{Pozostałe -- zaplanowane w punktach \ref{doskonalenie} na stronie \pageref{doskonalenie} oraz \ref{dzeduk} na stronie \pageref{dzeduk} \textit{Planu Rozwoju}}.
\end{enumerate}
\subsubsection{JAZ -- język angielski zawodowy}
W okresie wrzesień 2013 - czerwiec 2016 pracowałem łącząc dwie dziedziny: przedmioty zawodowe w klasach drugich i trzecich oraz język angielski zawodowy w klasach trzecich i~czwartych Technikum Łączności. W pierwszym roku pracy kontynuowałem zastany podręcznik ,,Infotech -- English for Computer Users", który okazał się zbyt wymagający (poziom B1--B2). Zaplanowałem zmianę podręcznika na ''English for IT2" (wyd. Pearson, poziom A2 -- B1). Po dłuższym namyśle ostatecznie wprowadziłem podręcznik ,,Vocational English" wydawnictwa Express Publishing. Nadmieniam, że przedmiot jest prowadzony w dwóch ostatnich klasach technikum, co prowadzi do konkluzji, że język ogólny prowadzony jest na bardzo niskim poziomie\footnote{patrz \url{/home/navegante/Documents/Technikum/VocationalEnglish/sprawozdanie\_voc\_eng.pdf}}.\newline
Główne działania w tej dziedzinie:
\begin{enumerate}
\begin{enumerate}
\item{Wprowadzenie odpowiedniego podręcznika}\newline
Podręcznik ,,Vocational English" wydawnictwa Express Publishing.
\item{Napisanie adekwatnego programu nauczania i jego realizacja}\newline
Program nauczania\footnote{\url{/home/navegante/Downloads/pdfs/Program\_nauczania\_angielski\_zawodowy.pdf}} p.t. ,,Program nauczania języka angielskiego w kształceniu zawodowym'' został napisany w czerwcu 2014 roku i~po zatwierdzeniu wprowadzony od września 2015 roku. 
\item{Intensywna praca na rzecz podniesienia poziomu nauczania i~skonkretyzowania wymagań}
\end{enumerate}
\end{enumerate}
\subsubsection{Zajęcia pozalekcyjne \textit{RoboTY}}\newline
Z początkiem roku szkolnego 2015 zaproponowałem założenie koła zainteresowań, którego celem jest budowa różnych modeli robotów, nauka sterowania urządzeń lub procesów za pomocą programowalnych kontrolerów logicznych PLC (\textit{programmable logic controllers}). Koło to o ambitnych założeniach ogólnych\footnote{\url{\~/Documents/Technikum/roboty/RoboTY.pdf}} podjęło działalność miesiąc później. Z~przerwą w pierwszym semestrze bieżącego roku szkolnego tj. 2016/2017 wciąż działa. Największym problemem w~jego działalności jest minimalne zainteresowanie uczniów Technikum Łączności. Dotychczasowe osiągnięcia zostały pokazane podczas dnia otwartego (w ramach promocji) Szkoły. Przygotowano trzy projekty (elektronika, program, montaż): sterowanie klasą szkolną (typu home automation), robot typu line-follower, bocznica kolejowa -- sterowanie ruchem zależnie od wielu zmiennych parametrów. Koło \textit{RoboTY} wiąże się ściśle z projektowanym w \textit{Planie Rozwoju} punktem \ref{kursPython} na stronie \pageref{kursPython} dotyczącym kursu programowania w języku Python. Program koła przewiduje naukę i~korzystanie z~programów w tym języku do~komunikacji z~robotami.
\subsubsection{Metody automatycznego testowania}\newline \label{autotest2}
Wymienione w \ref{autotest} na stronie \pageref{autotest} to trudny i dla informatyka jakże ciekawy temat. Praktycznie najłatwiej realizowalny w postaci autosprawdzalnych testów -- przykładem testy na platformie Moodle np. \textbf{Test\_partycjonowanie}\footnote{\url{http://moodle.zs37.waw.pl/mod/quiz/view.php?id=1927}}. Łatwo realizowalne w przypadku odpowiedzi zamkniętych. Podejmowałem również próby mające na celu ułatwienie sprawdzania prac zawierających pytania otwarte np. poprzez 
\begin{enumerate}
\begin{enumerate}
\item{sprawdzenie czy odpowiedź lub praca w wersji elektronicznej została oddana},
\item{automatyczne przygotowanie przeglądu sprawdzanych prac poprzez np. samoczynne uruchamianie odpowiedniego programu do otwarcia pliku i~zamykanie po~dokonaniu oceny\footnote{\url{/home/navegante/spr\_ubu.sh}},
\item{zliczanie punktów i wystawianie oceny zgodnie z WSO\footnote{\url{/home/navegante/Documents/Documents/procent\_ocena.py}} lub \footnote{\url{/home/navegante/python/ocena.py}}.
\end{enumerate}
\end{enumerate}

\subsubsection{Materiały edukacyjne na szkolnym serwisie Moodle}\newline
Dostosowując nauczane treści do warunków i możliwości uczniów intensywnie wykorzystuję szkolny serwis \textit{Moodle} \footnote{\url{http://moodle.zs37.waw.pl}} gdzie sukcesywnie umieszczam materiały edukacyjne, zadania i ćwiczenia.\newline 
Do najważniejszych należy skrypt, który został przygotowany w pierwszym semestrze mojej pracy w Technikum, będący swoistym podręcznikiem w wersji elektronicznej uzupełniającym książki do przedmiotu (tytuły). Wymienione podręczniki stanowią kompendium wiedzy na temat systemu operacyjnego Linux, który jest moją specjalnością, lecz brakuje w nich zadań typowo pracownianych. To właśnie było motywacją do przygotowania własnych treści.\newline
Ponieważ umieszczone na szkolnym \textit{Moodle} wymagają ciągłej aktualizacji, na początku kwietnia 2016 r. postanowiłem zbierać i aktualizować je w postaci książkowej. Obecnie materiał ten jako ,,nskrypt" umieszczony jest w serwisie \textit{github} \footnote{\url{https://github.com/zs37bojanowski/nskrypt}}. Materiał ten jest w~ciągłym rozwoju. Fragmenty tego podręcznika są opublikowane dla uczniów na szkolnym \textit{Moodle} w~formacie pdf.
Na tej platformie ponadto umieszczane są testy autosprawdzalne o czym więcej w punkcie: \ref{autotest2}.
\subsubsection{Doskonalenie warsztatu i metod pracy}\newline
W tych ramach przygotowałem metodykę klonowania zasobów systemowych stacji roboczych. Jednocześnie technika ta ma duży wpływ na sprawny przebieg sesji egzaminów zawodowych. W~ramach przygotowania uczniów do~egzaminów zawodowych E12 pod koniec maja 2017 r. zespół nauczycieli przedmiotów zawodowych przeprowadził próbny egzamin zawodowy E12. Korzystając z~okazji postanowiłem ułatwić sobie i~kolegom trudne zadanie stawiane przed tzw.~asystentem technicznym, którego zadaniem jest przygotowanie sali egzaminacyjnej do kolejnej sesji. Ze~względu na krótki czas między sesjami egaminacyjnymi zadanie to~jest bardzo stresujące. Zaproponowane przeze mnie rozwiązanie, przygotowane i testowane w maju 2017, oparte na~serwerze Clonezilla\footnote{\url{www.clonezilla.org}} dobrze sprawdziło się podczas sesji próbnych co stanowi dobrą prognozę na czas prawdziwych egzaminów. Napisałem krótką instrukcję użytkowania, która została przesłana zainteresowanym (załączyć w biblio).
\subsubsection{The last but not least}\newline
Podczas trwania stażu miały miejsce wydarzenia jednorazowe aczkolwiek angażujące czas i~kosztujące wiele wysiłku. Do takich właśnie należało przygotowanie i prowadzenie tygodniowego cyklu zajęć dla uczniów ze szkoły technicznej o podobnym do naszego profilu z~Ostrawy w~Czechach. Zajęcia te odbyły się w drugiej połowie września 2014 roku i~były poświęcone konsoli jako optymalnemu środowisku pracy: ,,The Console and Why It Is an Optimal Tool for Every day Use"\footnote{\url{/home/navegante/Documents/Technikum/czechlinux/ceskylinux.pdf}}. Odbywały się w języku angielskim.
\subsection{Przygotowanie sprawozdania z realizacji~planu rozwoju zawodowego.}
\subsection{Zakończenie realizacji~planu rozwoju zawodowego.}
%\bibliography{}
\end{flushleft}
\end{document}
%
